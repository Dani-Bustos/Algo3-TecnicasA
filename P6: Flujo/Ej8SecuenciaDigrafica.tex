\documentclass{article}
\usepackage{amsmath}

\begin{document}

\section*{Definición y Algoritmo para el Problema de Realización de Secuencia Digráfica}

\begin{enumerate}
  \item Dado un ordenamiento \( v_1, \ldots, v_n \) de los vértices de un digrafo \( D \), se define la secuencia digráfica de \( D \) como \( (d^-(v_1), d^+(v_1)), \ldots, (d^-(v_n), d^+(v_n)) \).
  \item Dada una secuencia de pares \( d \), el problema de realización de \( d \) consiste en encontrar un digrafo \( D \) cuya secuencia digráfica sea \( d \).
\end{enumerate}

\section*{Descripción del Algoritmo de Flujo Máximo}

\begin{itemize}
  \item Definimos dos conjuntos \( v_i^- \) y \( v_i^+ \).
  \item En el grafo, las relaciones existen solo entre conjuntos. Las relaciones forman un grafo casi completo donde existe la arista \( (v_i^+, v_j^-) \) para todo \( j \neq i \) con capacidad 1.
  \item Todos los vértices \( v_i^+ \) están conectados a una fuente \( s \) con capacidad \( d^+(v_i) \). Análogamente, todos los vértices \( v_i^- \) están conectados al sumidero con capacidad \( d^-(v_i) \).
  \item Se utiliza un algoritmo de flujo máximo. El sistema tiene solución si todas las aristas tanto de salida como de entrada están saturadas.
  \item Si hay flujo por una arista \( (v_i^+, v_j^-) \), entonces en nuestro digrafo solución existe la arista \( (v_i, v_j) \).
\end{itemize}





\section*{Complejidad del Algoritmo}

\begin{itemize}
  \item \( |V| := n + 2 \)
  \item \( |E| := 2n + \sum_{i=1}^{n-1} (n - 1) = \frac{n^2}{2} + \frac{3n}{2} \)
  \item Observemos que en el peor caso, el flujo máximo será \( \sum_{i=1}^{n} (n - 1) = O(n^2) \).
  \item Usando el algoritmo de Edmonds-Karp, la complejidad es \( O(\min(|V||E|^2, |V| \cdot F)) = O(\min(n^4, n^3)) = O(n^3) \).
\end{itemize}

\end{document}
