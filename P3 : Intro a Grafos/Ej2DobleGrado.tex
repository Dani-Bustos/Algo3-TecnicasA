\documentclass{article}
\usepackage{amsmath}
% Title
\title{Ejercicio 2: Doble Grafo}
\author{Daniel Bustos}
\date{27/4/2024}

\begin{document}
\maketitle
Sea $G = (E,V)$ un grafo con $|V| \geq 2$ . Supongamos que todos los vértices tienen distinto grado:

\[
\forall v \in V, w \in V , v \neq w \rightarrow d(v) \neq d(w)
\]

Luego podemos ordenar los vértices según su grado de manera creciente. Sea $n = |V|$, esto queda:

\[
d(v_1) < d(v_2) < d(v_3) < \ldots < d(v_n)
\]

Dado que no hay aristas duplicadas, tenemos que el grado de $d(v_n)$ es como máximo $n - 1$, luego:

\[
d(v_1) < d(v_2) < d(v_3) < \ldots < d(v_n) \leq n-1
\]

\textbf{Observación:} Si $d(v_n) < n-1$, debe haber al menos un repetido, ya que el grado es siempre mayor o igual a 0. Entonces vale que $d(v_n) = n-1$. Luego $v_n$ es un vecino universal (está conectado con todos). \\ \\
Dado que son todos de distintos grados, con el mayor $n-1$ y el menor $0$, vale que $d_i = i-1$.\\

Por lo tanto, $d(v_1) = 0$ y es un vecino aislado (no tiene conexiones). Pero $v_n$ era un vecino universal. Tenemos un nodo que esta conectado con todos y otro nodo que no esta conectado con ninguno. ¡Absurdo!\\

Por lo tanto, para todo grafo $G$, con mas de un vertice, existen dos vértices distintos con el mismo grado.

\end{document}
