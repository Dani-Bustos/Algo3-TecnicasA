\documentclass{article}
\usepackage{amsmath} % for mathematical symbols
\usepackage{lipsum} % for dummy text

\begin{document}

Sea $G = (V,E)$ un grafo. Queremos demostrar que si $G$ tiene más de $2n$ vértices con más de $n^2$ aristas, entonces tiene algún triángulo, formalmente tres vértices entre los cuales podemos armar un ciclo simple.\\

Demostremoslo por inducción:\\

Sea $P(n)$ := $|V| = 2n$ $\land$ $|E| > n^2$ $\rightarrow$ $\exists$ un triángulo.

\textbf{Caso base $n = 1$:}

Si $n = 1$, entonces hay $2$ vértices, con como máximo $1$ arista entre ellos. Luego se falsea el antecedente y vale $P(1)$.

\textbf{Paso inductivo:}

H.I: $\exists n_0$ tal que sea $G = (V,E)$ un grafo, si $|V| = 2n$ $\land$ $|E| > n_0^2 \rightarrow \exists$ un triángulo.

Q.V.Q: $P(n_0) \rightarrow P(n_0 + 1)$

Si el grafo no cumple el antecedente, $P(n_0 + 1)$ vale trivialmente.

Supongamos que lo cumple. Entonces $|V| = 2n_0 + 2$ $\land$ $|E| > (n_0 + 1)^2 = n_0^2 + 2n_0 + 1$.

La idea ahora va a ser sacar dos vértices para poder llegar a nuestra hipótesis inductiva, cumpliendo las condiciones.

Si sacamos dos vértices (Llamémoslos $v$ y $w$), la cantidad de aristas que tendremos será al menos: 

\[
|E| - d(v) - d(w) \geq n_0^2 + 2n_0 + 1  - d(v) - d(w) + 1 \quad (\text{ 1 + porque lo cambiamos a mayor igual})
\]

Hasta qué grados pueden tener $v$ y $w$ tal que se cumpla lo que queremos ver? :

\[
n_0^2 + 2n_0 + 2 - d(v) - d(w) > n_0^2  \leftrightarrow  2n_0 + 2 > d(v) + d(w)
\]

Entonces, tenemos que si existen dos nodos que cumplan esta propiedad en nuestro grafo original, entonces vale la H.I, haciendo valer el Paso inductivo.


Ahora me encantaria definir que quiere decir que este no pase, y decir: Bueno pero si esto no ocurre, hay siempre un triangulo por la cantidad de aristas que hay. Se aceptan sugerencias para como seguir la demo rigurosamente.


\end{document}
