\documentclass{article}
\usepackage{amsmath} % for mathematical symbols
\usepackage{lipsum} % for dummy text
% Title
\author{Daniel Bustos}
\date{3 de mayo de 2024}
\title{Triangulo Inductivo}

\begin{document}

\maketitle
Sea $G = (V,E)$ un grafo. Queremos demostrar que si $G$ tiene más de $2n$ vértices con más de $n^2$ aristas, entonces tiene algún triángulo, formalmente tres vértices entre los cuales podemos armar un ciclo simple.\\

Demostremoslo por inducción:\\

Sea $P(n)$ := $|V| = 2n$ $\land$ $|E| > n^2$ $\rightarrow$ $\exists$ un triángulo en G.\\

\textbf{Caso base $n = 1$:} \\

Si $n = 1$, entonces hay $2$ vértices, con como máximo $1$ arista entre ellos. Luego se falsea el antecedente y vale $P(1)$.\\

Extra : El primer caso donde vale es n = 2. Con 4 vertices y 5 aristas al menos, es facil imaginarse el triangulo\\
\textbf{Paso inductivo:}\\

H.I: $\exists n_0$ tal que sea $G = (V,E)$ un grafo, si $|V| = 2n_0$ $\land$ $|E| > n_0^2 \rightarrow \exists$ un triángulo.

Q.V.Q: $P(n_0) \rightarrow P(n_0 + 1)$ \\

Si el grafo no cumple el antecedente, $P(n_0 + 1)$ vale trivialmente.

Supongamos que lo cumple. Entonces $|V| = 2n_0 + 2$ $\land$ $|E| > (n_0 + 1)^2 = n_0^2 + 2n_0 + 1$.

La idea ahora va a ser sacar dos vértices para poder llegar a nuestra hipótesis inductiva, cumpliendo las condiciones que esta nos pida.

Si sacamos dos vértices (Llamémoslos $v$ y $w$), la cantidad de aristas que tendremos será al menos: 

\[
|E| - d(v) - d(w) \geq n_0^2 + 2n_0 + 1  - d(v) - d(w) + 1 \quad (\text{ 1 + porque lo cambiamos a mayor igual})
\]

¿Hasta qué grados pueden tener $v$ y $w$ tal que se cumpla lo que queremos ver? :

\[
n_0^2 + 2n_0 + 2 - d(v) - d(w) > n_0^2   
\]
\[
 \leftrightarrow  2n_0 + 2 > d(v) + d(w)
\]

Entonces, tenemos que si existen dos nodos que cumplan esta propiedad en nuestro grafo original, entonces vale la H.I, haciendo valer el Paso inductivo.
Veamos ahora el otro caso.\\ \\
Supongamos que para todos dos nodos que agarremos, esta desigualdad no vale. Veamos que en este caso, deben haber dos nodos que compartan un vecino, generando así un triángulo. Formalmente:\\

En un grafo \( G = (V,E) \), con \( |V| = 2n_0 + 2 \) y \( |E| > (n_0 + 1)^2 \)

\[
 \forall v,w \in V \ / \ 2n_0 + 2 \leq d(v) + d(w)  \rightarrow  \ \exists v',w' \in V :   v' \neq  w' \ \land \  N(v') \cap N(w') \neq \emptyset
\]

Con N(v) y N(w) el vecindario de v y w respectivamente (sus nodos adyacentes)\\
Probemos por el contrarrecíproco la verdad de esta afirmación:

Tomemos \( v_0 \) y \( w_0 \) dos vértices cualesquiera, sabemos que no comparten vecinos.

Si no comparten vecinos, necesariamente debe valer que \( d(v_0) + d(w_0) \leq 2n_0 + 2 \) ya que la totalidad del grafo tiene \( 2n_0 + 2 \) nodos. De otra manera, nuestro grafo tendría al menos \( 2n_0 + 2 + 2 \) vértices, lo cual nos lleva a un absurdo.

Por lo tanto, vimos que si no vale la desigualdad original para cualesquiera dos vertices, entonces hay al menos dos vértices con un vecino en común, generándonos así un triángulo. 

Entonces \( P(n_0 + 1) \) vale para cualquier caso.\\

Q.E.D vale \( P(n) \ \forall n \geq 1 \)


\end{document}
