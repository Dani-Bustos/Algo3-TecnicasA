\documentclass{article}
\usepackage[spanish]{babel}
\usepackage{amsmath}

\title{Unicidad Digrafo}
\author{Daniel Bustos}
\date{27 de abril de 2024}

\begin{document}
\maketitle

Sea $G = (E,V)$ un grafo orientado, con cada vértice un grado distinto de salida. Formalmente:
\[
\forall v \in V, w \in V, v \neq w \Rightarrow d_{\text{out}}(v) \neq d_{\text{out}}(w)
\]

Podemos ordenarlos por orden creciente:
\[
d_{\text{out}}(v_1) < d_{\text{out}}(v_2) < \ldots < d_{\text{out}}(v_n)
\]

OBS: Como el grado máximo de salida es $n - 1$ y el mínimo $0$, como todos deben ser distintos, vale que $d_{\text{out}}(v_i) = i - 1$.

Supongamos ahora la existencia de un $G' = (E',V')$, $|V'| = n$ grafo, que cumple las mismas propiedades. Supongamos que NO es isomorfo a $G$.
\end{document}
