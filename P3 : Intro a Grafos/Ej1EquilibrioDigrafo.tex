\documentclass{article}
\usepackage{amsmath}

\begin{document}

\title{Equilibrio Digrafo}
\author{Daniel Bustos}
\date{27/4/2024}
\maketitle

Sea \( D = (E,V) \) un grafo.

Queremos probar que 

\[
\sum_{v \in V} d_{\text{in}}(v) = \sum_{v \in V} d_{\text{out}}(v) = |E(D)|.
\]

Probemos por inducción. Sea \( P(n) \) la propiedad anterior con \( n = |E(D)| \).

\textbf{Caso base:} \( |E(D)| = 0 \). Como no hay aristas, vale la propiedad:

\[
\sum_{v \in V} d_{\text{in}}(v) = \sum_{v \in V} d_{\text{out}}(v) = |E(D)| = 0.
\]

\bigskip

\textbf{Paso inductivo:} Hipótesis Inductiva (H.I): Vale que si \( |E(D)| \leq n_0 \), entonces vale \( P(n) \).

Si \( |E(D)| = n_0 + 1 \), podemos tomar una arista cualquiera, sacársela y llamemos \( D' \) al grafo resultante. Luego \( |E(D')| = n_0 \). Por lo tanto, vale la H.I: 

\[
\sum_{v \in V} d_{\text{in}}(v) = \sum_{v \in V} d_{\text{out}}(v) = |E(D')|.
\]

Observemos que al ser un grafo dirigido, sacar una arista implica restar uno a la suma de los grados de entrada y restar uno a la de salida, luego vale que:

\[
\sum_{v \in V} d_{\text{in}}(v) + 1 = \sum_{v \in V} d_{\text{out}}(v) + 1 = |E(D')| + 1 = |E(D)|.
\]

Que era lo que queríamos ver, luego \( P(n) \) vale para todo \( n \geq 0 \).

\end{document}
