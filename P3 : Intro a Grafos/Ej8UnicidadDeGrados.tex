\documentclass{article}
\usepackage[utf8]{inputenc}
\usepackage{amsmath}
\usepackage{amsthm} % For theorems and proofs
\usepackage{tikz}

\author{Daniel Bustos}
\title{Unicidad Digrafos}
\date{30/04/2024}
\begin{document}

\maketitle


Sean $G_2 = K_2$ y $G_{n+1} = \overline{G_n \cup K_1}$ para todo $n \geq 2$. Queremos demostrar que $G_n$ tiene un único par de vértices de igual grado. (Denotamos $K_i$ para referirnos al grafo completo de $i$ vértices)

Probémoslo por inducción:\\

Sea $P(n)$ := El grafo $G_n$ como definido anteriormente tiene un único par de vértices de igual grado.\\

\textbf{Caso base $n = 2$:}\\

El completo de dos vértices es de la siguiente forma:
\begin{tikzpicture}
  % Vertices
  \node[circle, draw, fill=black, inner sep=1pt, label=left:$1$] (1) at (0,0) {};
  \node[circle, draw, fill=black, inner sep=1pt, label=right:$2$] (2) at (2,0) {};

  % Edges
  \draw (1) -- (2);
\end{tikzpicture} \\

Solo hay 2 vértices, cada uno con grado 1. Vale $P(2)$.\\

\textbf{Paso inductivo:}

H.I: Existe $n$ tal que el grafo $G_n$ tiene un único par de vértices de igual grado. Mostraremos $P(n) \rightarrow P(n+1)$.

Sabemos por la definición original que:
\[ G_{n+1} = \overline{G_n \cup K_1} \]

¿Cómo afecta al grado de cada vértice el conjugado? Observemos que para cada vértice $v_i$, con grado $d(v_i)$, al momento de tomar el conjugado el vértice perderá sus conexiones preexistentes y ganará tantas como no tenía antes respecto de los $n$ vértices más  una arista adicional por la conexión con el único vértice de $K_1$. En otras palabras:
\[ \overline{d(v_i \ \cup\  K_1)} = n - 1 -d(v_i) + 1 = n - d(v_i) \]

A partir de esta propiedad se observa que dos vértices $w$ y $v$ tienen la misma cantidad de vértices al conjugar si solo si:
\[ \overline{d(w \ \cup\  K_1)} = \overline{d(v \ \cup\  K_1)} \  \leftrightarrow \ n - 1 - d(v) = n - 1 - d(w) \  \leftrightarrow  \ d(w) = d(v) \]

Esto solo puede pasar si $w_i = v_i$ o su grado era el mismo antes del conjugado. Se separa en dos casos:\\

1) Si no hay ningún vértice en $G_n$ que tenga grado 0, por H.I hay solo 2 vértices que tienen el mismo grado y estos mismos serán los únicos que seguirían teniendo mismo grado luego de realizar la operación $\overline{G_n \cup K_1}$ por lo recien visto. \\

2) Si hay un vértice de $G_n$ que tenga grado 0, ya no vale $P(n + 1)$ ya que ,por lo visto anteriormente, el vértice de $K_1$ (de grado 0 originalmente) tendrá grado $n$ luego de conjugar, lo mismo para cualquier otro vértice de grado 0 en $G_n$, además de que tenemos la pareja adicional que nos da la H.I.\\

Veamos ahora por otra inducción que no es posible que haya vértices en $G_n$ con grado 0.\\

Sea $P'(n)$ := No hay vértices de grado 0 en $G_n$ para todo $n \geq 2$.\\

\textbf{Caso base, $n = 2$:}

Para este caso $G_2 = K_2$ y el completo 2 tiene solo dos vértices de grado 1. Luego vale $P'(2)$.\\

\textbf{Paso inductivo:}

H.I: Existe un $n$ tal que vale que no hay vértices de grado 0 en $G_n$. \\ 
Mostraremos $P'(n) \rightarrow P'(n+1)$.\\

Sabemos que $G_{n+1} = \overline{G_n \cup K_1}$ y el grado de cada vértice luego de conjugar en esta unión está dado por la fórmula:
\[ \overline{d(v_i \ \cup\  K_1)} =  n - d(v_i) \]

Veamos que no puede ser 0 el grado:
\[ \overline{d(v_i \ \cup\  K_1)}  = 0 \leftrightarrow n - d(v_i) = 0 \leftrightarrow n = d(v_i) \]

Esto es absurdo ya que los vértices que pertenecían a $G_n$ podían tener grado $n-1$ como máximo, y el unico vértice  perteneciente a $K_1$ tiene siempre grado 0 inicialmente. Luego vale el paso inductivo.

Vimos que vale $P'(n)$ para todo $n \geq 2$. Como esto nos dice que el caso 2 en la demostración de $P(n+1)$ es imposible, vale el paso inductivo de $P(n)$.

Por lo tanto, vale lo que queríamos demostrar originalmente.

\end{document}
