\documentclass{article}
\usepackage{amsmath}
\author{Daniel Bustos}
\title{Distancias de Secuencias}

\begin{document}
\maketitle
10. \\
 Se define la distancia entre dos secuencias de naturales \( X = x_1, \ldots, x_k \) e \( Y = y_1, \ldots, y_k \) como
\[
d(X, Y) = \sum_{i=1}^k |x_i - y_i|.
\]
Dado un conjunto de secuencias \( X_1, \ldots, X_n \), cada una de tamaño \( k \), su grafo asociado \( G \) tiene un vértice \( v_i \) por cada \( 1 \leq i \leq n \) y una arista \( v_i v_j \) de peso \( d(X_i, X_j) \) para cada \( 1 \leq i < j \leq n \). Proponer un algoritmo de complejidad \( O(kn^2) \) que, dado un conjunto de secuencias, encuentre el árbol generador mínimo de su grafo asociado.

Idea del algoritmo:

Vamos a construir el grafo en el cual todos están conectados con todos, y el peso de cada arista \( (X, Y) \) es \( d(X, Y) \). ¿Cuánto nos cuesta construirlo? Dado que computar \( d(X, Y) \) cuesta \( O(k) \) y hay \( \binom{n}{2} \) aristas, el costo total de construcción del grafo es
\[
k \cdot \binom{n}{2} \in O(n^2 k).
\]
Luego utilizamos el algoritmo de Prim con costo de \( O(n^2) \) para generar el árbol mínimo, dejándonos un costo total de \( O(n^2 k + n^2) \), que se incluye en \( O(n^2 k) \).

\end{document}
