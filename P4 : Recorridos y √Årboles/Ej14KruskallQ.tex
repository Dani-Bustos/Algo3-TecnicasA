\documentclass{article}
\usepackage{amsmath}

\begin{document}

\section*{14. Kruskal Prioridad Q}

Sea $q : V(G) \rightarrow \mathbb{Z}$ una función inyectiva para un grafo $G$. Demostrar que $G$ tiene un único árbol generador mínimo si y solo si el algoritmo de Kruskal con prioridad $q$ retorna el mismo árbol que el algoritmo de Kruskal con prioridad $-q$.

\textbf{IDA:}

Si tiene un único árbol generador mínimo, el algoritmo de Kruskal retorna siempre ese árbol, indistinto de la prioridad.

\textbf{Vuelta:}

El algoritmo de Kruskal con prioridad $q$ retorna el mismo árbol que el algoritmo de Kruskal con prioridad $-q$ $\implies G$ tiene un único árbol generador mínimo.

Probemos que esto quiere decir que tiene todos los pesos distintos. Para así usar el inciso b del ejercicio anterior (Si todos los pesos son distintos entonces hay un único AGM).

Supongamos que hay dos iguales, llamémoslas $a$ y $b$ con $q(a) > q(b)$. En la prioridad $q$, Kruskal siempre toma $b$ en la primera iteración. Luego, si usamos $-q$, Kruskal toma $b$, pero como los árboles resultantes son iguales, necesariamente $a$ y $b$ son tomados en ambos ordenes. 
\end{document}