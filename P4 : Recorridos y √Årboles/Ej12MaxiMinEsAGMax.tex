\documentclass{article}
\usepackage[utf8]{inputenc}
\usepackage{amsmath}
\usepackage{amssymb}

\title{Árbol Maximin es Arbol Generador Maximo}
\author{Daniel Bustos}

\begin{document}

\maketitle

\textit{Demostrar que \(T\) es un árbol maximin de \(G\) $\leftrightarrow$ \(T\) es un árbol
generador máximo de \(G\). Concluir que todo grafo conexo \(G\) tiene un árbol maximin que puede
ser computado con cualquier algoritmo para computar árboles generadores máximos.}

\textbf{\(T\) es un árbol maximin de \(G\) \(\implies\) \(T\) es un AGMax}

Sabemos que \( \text{bwd}_T(v,w) = \text{bwd}_G(v,w) \ \forall \ v,w \in V(G)\), con \(\text{bwd}(v,w)\) el máximo mínimo peso entre todos los posibles caminos dentro del grafo entre \(v\) y \(w\).

Q.V.Q. \(T\) es AGMax, es decir, la suma de sus pesos es máxima y genera a \(G\).

Sea \(T'\) el AGMax que tenga más aristas en común con \(T\). Por ser \(T\) y \(T'\) árboles, ambos tienen \(n-1\) aristas.

Sea $vw$ una arista en $T$ que no está en $T'$ y llamemos $P$ al único camino que conecta a $v$ y $w$ en $T$. Sea $xy$ la arista de menor peso en $P$. 

Sabemos que por ser maximin, todo camino tiene necesariamente menor o igual valor mínimo entre todo par de vértices. Consideremos el árbol $T'-xy$, que necesariamente tiene dos partes conexas. Si ahora le agregamos la arista $vw$, volvemos a formar un árbol conexo, ya que esta arista nos une dos elementos en particiones distintas, y tenemos nuevamente $n-1$ aristas. Miremos el costo de $T'-\{xy\} \cup \{vw\}$. Ahora vale que:

\[
c(T') - c(xy) + c(vw) \leq c(T') \leftrightarrow c(vw) \leq c(xy)
\]

Consideremos ahora los dos casos:

\begin{itemize}
    \item Si $c(vw) < c(xy)$: Tenemos un absurdo, ya que $xy$ es la arista de menor peso en el camino que conecta a $v$ y $w$ en $T$. Como $vw$ era parte de un camino en un árbol maximin, tenemos que necesariamente $c(xy) < c(vw)$.
    \item Si $c(vw) = c(xy)$: Entonces tenemos que este nuevo árbol $T'-\{xy\} \cup \{vw\}$ es AGMmax, pero esto es absurdo ya que comparte más aristas con $T$ que $T'$ y es AGMmax. Luego el absurdo provino de suponer que existía una arista distinta, dándonos que $T$ es AGMax por ser minimax.
\end{itemize}




\end{document}
