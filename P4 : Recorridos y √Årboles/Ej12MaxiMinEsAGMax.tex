\documentclass{article}
\usepackage[utf8]{inputenc}
\usepackage{amsmath}
\usepackage{amssymb}

\title{Árbol Maximin es Arbol Generador Maximo}
\author{Daniel Bustos}

\begin{document}

\maketitle

\textit{Demostrar que \(T\) es un árbol maximin de \(G\) $\leftrightarrow$ \(T\) es un árbol
generador máximo de \(G\). Concluir que todo grafo conexo \(G\) tiene un árbol maximin que puede
ser computado con cualquier algoritmo para computar árboles generadores máximos.}

\textbf{\(T\) es un árbol maximin de \(G\) \(\implies\) \(T\) es un AGMax}

Sabemos que \( \text{bwd}_T(v,w) = \text{bwd}_G(v,w) \ \forall \ v,w \in V(G)\), con \(\text{bwd}(v,w)\) el máximo mínimo peso entre todos los posibles caminos dentro del grafo entre \(v\) y \(w\).

Q.V.Q. \(T\) es AGMax, es decir, la suma de sus pesos es máxima y genera a \(G\).

Sea \(T'\) el AGMax que tenga más aristas en común con \(T\). Por ser \(T\) y \(T'\) árboles, ambos tienen \(n-1\) aristas. Sea \(l'\) una arista tal que \(l' \in E(T') \land l' \notin E(T)\). Como ambos tienen la misma cantidad de aristas, 
\(\exists \ l\) arista tal que \(l \in E(T) \land l \notin E(T')\).

\end{document}
