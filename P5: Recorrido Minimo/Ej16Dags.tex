\documentclass{article}
\usepackage{amsmath}
\usepackage{amssymb}
\usepackage{algorithm}
\usepackage{algorithmic}

\begin{document}

\section*{16. Sea D un digrafo conexo que no tiene ciclos dirigidos, v el único vértice de D con grado de entrada 0 (Dicho vértice siempre existe, ver guías 2 y 3) y $c : E(D) \rightarrow \mathbb{Z}$ una función de pesos.}

\textit{a. Definir una función recursiva $d: V(D) \rightarrow \mathbb{Z}$ tal que $d(w)$ es el peso del camino mínimo de v a w para todo $w \in V(D)$. Ayuda: considerar que el camino mínimo de v a w se obtiene yendo de v hacia z y luego tomando la arista $z \rightarrow w$, para algún vecino de entrada z de w; notar que la función recursiva está bien definida porque D no tiene ciclos.}

\[
d(w) = \begin{cases} 
0 & \text{si } w = v \\
\min_{\text{vecino } z \in N^-(w)} \left( d(z) + c(z, w) \right) & \text{si } w \neq v 
\end{cases}
\]

\textit{b. Diseñar un algoritmo de programación dinámica top-down para el problema de camino mínimo en digrafos sin ciclos y calcular su complejidad.} \\

Como nuestra función tiene como máximo $n$ argumentos (uno por cada vértice en el grafo) y claramente hay superposición de subproblemas, entonces podemos usar programación dinámica.

Construimos un vector de $n$ posiciones:

\[
d(w) = \begin{cases}
\text{memo}[w] & \text{si memo}[w] \neq \text{INDEF} \\
0 & \text{si } w = v \\
\min_{\text{vecino } z \in N^-(w)} \left( d(z) + c(z, w) \right) & \text{si } w \neq v
\end{cases}
\]

La complejidad es $O(n +m )$, ya que recorremos todos los nodos y aristas. \\

\textit{c. (Integrador y opcional) Diseñar un algoritmo de programación dinámica bottom-up para el problema. Ayuda: computar $d$ de acuerdo a un orden topológico $v = v_1, \ldots, v_n$ donde $v_i \rightarrow v_j$ solo si $i < j$. Este orden se puede computar en $O(n + m)$ (guía 3).}

\begin{algorithm}
\caption{Computación de caminos mínimos en un digrafo acíclico (bottom-up)}
\label{alg:min_paths}

\begin{algorithmic}[1]
\STATE Computar un orden topológico $v = v_1, \ldots, v_n$ del grafo $D$
\STATE Inicializar vector $d$ con distancias infinitas: $d[w] = \infty$ para todo $w \in V(D)$
\STATE $d[v] = 0$
\FOR{$i = 1$ \TO $n$}
    \FOR{cada vecino $z$ de $v_i$ en $N^-(v_i)$}
        \STATE $d[v_i] = \min(d[v_i], d[z] + c(z, v_i))$
    \ENDFOR
\ENDFOR
\end{algorithmic}

\end{algorithm}

\end{document}
