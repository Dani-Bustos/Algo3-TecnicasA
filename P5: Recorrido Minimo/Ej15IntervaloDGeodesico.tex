\documentclass{article}
\usepackage[utf8]{inputenc}
\usepackage{amsmath}
\usepackage{algorithm}
\usepackage{algorithmic}

\begin{document}

\section*{Algoritmo para determinar si un conjunto de vértices es geodésico}
\textit{15. Dados dos vértices \( v \) y \( w \) de un grafo pesado \( G \), el intervalo entre \( v \) y \( w \) es el conjunto \( I(v, w) \) que contiene a todos los vértices que están en algún recorrido mínimo entre \( v \) y \( w \). Un conjunto de vértices \( D \) es geodésico cuando \( \bigcup_{(v,w)} I(v,w) = V(G) \). Diseñar e implementar un algoritmo de tiempo \( O(n^3) \) que, dado un grafo pesado y conexo \( G \) y un conjunto de vértices \( D \) de \( G \), determine si \( D \) es geodésico.}
\begin{algorithm}
\caption{Verificar si un digrafo es geodésico}
\label{alg:geodesic_check}

\begin{algorithmic}[1]
\STATE Utilizar el algoritmo de Floyd-Warshall para encontrar las distancias más cortas entre todos los pares de vértices y mantener la matriz de padres $\text{padre}[i][j]$, donde $\text{padre}[i][j]$ es el vértice precedente de $j$ en un camino más corto desde $i$ a $j$.
\STATE Inicializar un conjunto $R$ como vacío.
\FOR{cada par de vértices $v, w \in D$}
    \STATE Recorrer la matriz de padres para el par $(v, w)$ para obtener todos los vértices por los que se pasa en algún camino más corto de $v$ a $w$.
    \FOR{cada vértice visitado en el camino más corto de $v$ a $w$}
        \STATE Añadir el vértice visitado al conjunto $R$.
    \ENDFOR
\ENDFOR
\STATE Verificar si $R$ es igual al conjunto de todos los vértices $V(G)$ del grafo $G$:
    \IF{$R = V(G)$}
        \STATE Entonces, $D$ es geodésico.
    \ELSE
        \STATE En caso contrario, $D$ no es geodésico.
    \ENDIF
\end{algorithmic}

\end{algorithm}

El tiempo de ejecución del algoritmo es \( O(n^3) \), donde \( n \) es el número de vértices en \( G \).

\end{document}
