\documentclass{article}
\usepackage{amsmath} % for mathematical symbols and environments
\usepackage{amssymb} % for \mathbb command

\begin{document}

\section*{Encontrar la arista st-eficiente en un digrafo sin ciclos de peso negativo}

Dado un digrafo \( D \) con pesos \( c : E(D) \rightarrow \mathbb{R} \) que no tiene ciclos de peso negativo, queremos encontrar la arista \( v \rightarrow w \) que sea st-eficiente para la mayor cantidad de pares \( s \) y \( t \). Proponer un algoritmo eficiente y simple de programar para resolver este problema. Ayuda: verificar que la propiedad del Ejercicio 1a también es cierta en este caso.

\subsection*{Algoritmo propuesto}

\begin{enumerate}
    \item Correr Floyd-Warshall sobre el grafo \(\ O(n^{3})\).
    \item Crear un vector \( \text{mVec} \) de aristas. \(\ O(m)\)
    \item Para cada arista \( (v,w) \in E(G) \):
    \begin{enumerate}
        \item Para cada par de nodos \( (s,t) \) en la matriz:
        \begin{enumerate}
            \item Si \( M[s,t] == M[s,v] + c(v,w) + M[w,t] \), entonces \( \text{mVec}[vw]++ \).
        \end{enumerate}
    \end{enumerate}
    \item Tomar la máxima arista del vector \( \text{mVec} \).
\end{enumerate}

\subsection*{Complejidad}

El algoritmo propuesto tiene una complejidad de \( O(\min(n^{2}m, n^{3})) \), donde \( n \) es el número de nodos y \( m \) es el número de aristas del grafo.

\end{document}
