\documentclass{article}
\usepackage{amsmath}

\begin{document}

\section*{2. Diseñar un algoritmo eficiente que, dado un digrafo \( G \) con pesos no negativos, dos vértices \( s \) y \( t \) y una cota \( c \), determine una arista de peso máximo de entre aquellas que se encuentran en algún recorrido de \( s \) a \( t \) cuyo peso (del recorrido, no de la arista) sea a lo sumo \( c \). Demostrar que el algoritmo propuesto es correcto.}

Uso Dijkstra desde \( s \), luego Dijkstra desde \( t \) sobre el grafo transpuesto.

Ahora definimos el conjunto \( A := \{(v,w) \in E(G) : d(s,v) + c(v,w) + d(w,t) \leq c\} \). Nuestro resultado es el máximo de ese conjunto.

\textbf{Correctitud::}

Demostremos que si una arista \( (v,w) \) $\notin$ \( A \), entonces ningún recorrido que haga puede cumplir la cota \( c \). En particular, como \( d(s,v) + c(v,w) + d(w,t) \) es el mínimo recorrido que va de \( s \) a \( t \) pasando por \( c(v,w) \), ya que \( d(s,w) \) es la mínima distancia hasta un extremo de la arista y \( d(w,t) \) es la mínima al otro extremo, si este tiene un costo mayor a \( c \), se sigue que todos los demás lo tendrán. Luego, el conjunto \( A \) define correctamente nuestras aristas candidatas.

\end{document}
