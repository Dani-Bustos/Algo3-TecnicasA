\documentclass{article}
\usepackage{amsmath}
\usepackage{amsfonts}
\usepackage{algorithm}
\usepackage{algorithmic}

\begin{document}

\section*{4. Aristas Eficientes, Daniel Bustos}

Sea \( G \) un digrafo con pesos positivos que tiene dos vértices especiales \( s \) y \( t \). Para una arista \( e \notin E(G) \) con peso positivo, definimos \( G + e \) como el digrafo que se obtiene de agregar \( e \) a \( G \). Decimos que \( e \) mejora el camino de \( s \) a \( t \) cuando \( d_G(s, t) > d_{G+e}(s, t) \). Diseñar un algoritmo eficiente que, dado un grafo \( G \) y un conjunto de aristas \( E \notin E(G) \) con pesos positivos, determine cuáles aristas de \( E \) mejoran el camino de \( s \) a \( t \) en \( G \). Demostrar que el algoritmo es correcto.

\subsection*{Idea del algoritmo}

Similar al ejercicio 3, usamos Dijkstra sobre \( s \) y luego sobre \( t \) en el grafo transpuesto. Luego vamos agregando las aristas una a una y nos fijamos si mejoran el camino usando que una arista \( vw \) es \( st \)-eficiente \(\leftrightarrow d(s,v) + c(v,w) + d(w,t)\). Si al poner alguna arista esto mejora la mejor distancia, entonces esa arista mejora el camino.

\subsection*{Algoritmo}

\begin{algorithm}[H]
\caption{Determinar qué aristas mejoran el camino de \( s \) a \( t \)}
\begin{algorithmic}[1]
\STATE Ejecutar Dijkstra desde \( s \) y desde \( t \) en el grafo transpuesto.

\vspace{0.5em}

\STATE $\text{distancia\_actual} \gets d(s, t)$ \COMMENT{Obtenido del Dijkstra anterior}

\vspace{0.5em}

\FOR{arista \( (v, w) \) \textbf{in} \( E \)}
    \vspace{0.5em}
    \IF{$d(s, v) + c(v, w) + d(w, t) < \text{distancia\_actual}$}
        \vspace{0.5em}
        \STATE Marcar \( (v, w) \) como una arista que mejora el camino
        \vspace{0.5em}
    \ENDIF
    \vspace{0.5em}
\ENDFOR

\vspace{0.5em}

\RETURN Todas las aristas que mejoran el camino de \( s \) a \( t \)

\end{algorithmic}
\end{algorithm}

\subsection*{Demostración}

Como \( G \) es un digrafo con pesos positivos, sabemos por la propiedad del problema 1 que una arista \( ab \) es \( s \)-\( t \) eficiente \(\leftrightarrow d(s, a) + c(a, b) + d(b, t) = d(s, t) \).

Sea \( G \) nuestro digrafo, con \( P \) un camino mínimo. Consideremos ahora \( G + e \). Sabemos por la propiedad 1 que si en \( G + e \), \( d(s, t) = d(s, e) + c(e) + d(e, t) = d(s, t) \), entonces necesariamente forma parte de un camino mínimo. Si esto se cumple sobre \(G + e\) Tenemos dos casos:

\begin{itemize}
    \item \( e \) no cambia el camino mínimo, en cuyo caso nuestro algoritmo comparará con \( d(s, t) \) en \( G \) y no tomará a \( e \) como eficiente.
    \item \( e \) mejora el camino mínimo anterior, luego nuestro algoritmo la marcará como eficiente.
\end{itemize}

\end{document}
