\documentclass{article}
\usepackage{amsmath}
\usepackage{amsfonts}
\usepackage{algorithm}
\usepackage{algorithmic}

\begin{document}

\section*{3. Una Arista Negativa , Daniel Bustos}

Diseñar un algoritmo eficiente que, dado un digrafo pesado \( G \) y dos vértices \( s \) y \( t \), determine el recorrido mínimo de \( s \) a \( t \) que pasa por a lo sumo una arista de peso negativo. Demostrar que el algoritmo propuesto es correcto.

\subsection*{Idea del algoritmo}

Usamos Dijkstra sobre el grafo sin pesos negativos, tanto desde \( s \) como desde \( t \) sobre \( G \) transpuesto. Obtenemos así las distancias \( d(s, v) \) y \( d(v', t) \forall v, v' \in V(G) \). Luego consideramos los pesos de los caminos que incluyen una arista negativa, definido como: El costo de llegar hasta ellas desde \( s \), sumado el costo de la arista negativa, y desde ellas hasta \( t \) es menor que nuestro mejor camino anterior. Observemos que si la única manera de llegar de \( s \) a \( t \) fuese pasar por más de una arista negativa, entonces la distancia final resultante sería infinito, ya que las distancias que obtenemos en Dijkstra solo consideran que había aristas positivas.

\subsection*{Algoritmo}

\begin{algorithm}[H]
\caption{Encontrar el recorrido mínimo de \( s \) a \( t \) con a lo sumo una arista de peso negativo}
\begin{algorithmic}[1]
\STATE Remover todas las aristas negativas del grafo \( G \).

\vspace{0.5em}

\STATE Ejecutar Dijkstra desde \( s \) y desde \( t \) en el grafo transpuesto.

\vspace{0.5em}

\STATE $\text{candidato} \gets d(s, t)$ \COMMENT{Obtenido del Dijkstra anterior}

\vspace{0.5em}

\FOR{cada arista \( (a, b) \) negativa en \( G \)}

    \vspace{0.5em}
    
    \IF{$d(s, a) + c(a, b) + d(b, t) < \text{candidato}$}
    
        \vspace{0.5em}
        
        \STATE $\text{candidato} \gets d(s, a) + c(a, b) + d(b, t)$
        
        \vspace{0.5em}
        
    \ENDIF
\ENDFOR

\vspace{0.5em}

\RETURN \text{candidato}

\end{algorithmic}
\end{algorithm}

\subsection*{Complejidad}

Sea \( k \) la cantidad de aristas negativas con \( k \leq m \). La complejidad del algoritmo es de: 
\[ n + m + 2\Theta(m + n \log n) + k \leq O(m + n \log n) \]

\subsection*{Demostración}

Es claro que nuestra solución es:
\[ \min(\text{Camino mínimo entre } s \text{ y } t \text{ sin negativos}, \min(\text{Camino con una arista negativa entre } s \text{ y } t)) \]

Nuestro algoritmo nos da:
\[ \min \left( \min(\text{Caminos entre } s \text{ y } t \text{ sin negativos}), \min(d(s, a) + c(a, b) + d(b, t) \mid (a, b) \text{ es una arista negativa}) \right) \]

Es trivial que el camino mínimo entre \( s \) y \( t \) sin aristas negativas es igual a \( \min(\text{Caminos entre } s \text{ y } t \text{ sin negativos}) \).

Para la segunda parte, llamemos 
\begin{itemize}
    \item \( P' := \{d(s, a) + c(a, b) + d(b, t) \mid (a, b) \text{ es una arista negativa}\} \)
    
    \item \( P \) := los caminos con una arista negativa entre \( s \) y \( t \).
\end{itemize}

Veamos que \( \min(P) = \min(P') \). Podemos definir al peso de todos los caminos negativos con una arista negativa entre \( s \) y \( t \) exactamente como definimos \( P' \). Luego, sus mínimos deben ser iguales.


\end{document}
