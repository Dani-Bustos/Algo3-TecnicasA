\documentclass{article}
\usepackage{amsmath} % for mathematical symbols and environments
\usepackage{amsfonts}
\title{Matriz de Floyd-Warshall en Grafos}
\author{Matías Waisman, Daniel Bustos}
\begin{document}


\maketitle
\textit{Decimos que una matriz cuadrada, simetrica y positiva \( M \in \mathbb{N}^{2} \) es de Floyd-Warsha ll (FW) si existe un grafo \( G \) tal que \( M \) es el resultado de aplicar FW a \( G \). Describir un algoritmo para decidir si una matriz \( M \) es FW. En caso afirmativo, el algoritmo debe retornar un grafo \( G \) con la mínima cantidad de aristas posibles tal que el resultado de FW sobre \( G \) sea \( M \). En caso negativo, el algoritmo debe retornar alguna evidencia que pruebe que \( M \) no es FW.} \\

A priori sabemos que si la matriz no es simétrica entonces no es de Floyd-Warshall sobre un grafo. En particular, si la matriz es FW, se debe cumplir que su diagonal sean todos \( 0 \) (ya que la distancia \( d(v,v) = 0 \ \forall v \in V \)). Además, debemos chequear que para todo par de vértices \( s,t \), debe ocurrir que:
\[
d(s,t) \leq d(s,w) + d(w,t) \quad \forall w \in V
\]
De otra forma, no sería un camino mínimo. Esta condición la podemos chequear en \( \Theta(n^{3}) \). Podemos devolver los 3 nodos que no cumplen esta condición en el caso de que los hubiese, demostrando que la matriz no es FW.

Si se cumplen todas estas condiciones, sabemos que tenemos una matriz FW. Para construir el camino mínimo haremos lo siguiente: 
\begin{itemize}
    \item Vamos a "construir" el grafo completo, con los costos como indica la matriz.
    \item Luego, para cada vértice \( v \), vamos a ver todos sus vecinos \( w \) y ver los vecinos de \( w \), llamados \( x \), y vamos a chequear si \( d(v,w) = d(v,x) + d(x,w) \).
    \item Si se cumple esto, podemos eliminar la arista que conecta \( v \) y \( w \).
\end{itemize}
Esto se puede resolver en complejidad \( \Theta(n^{3}) \), con una implementación muy similar a Floyd-Warshall.

\end{document}
