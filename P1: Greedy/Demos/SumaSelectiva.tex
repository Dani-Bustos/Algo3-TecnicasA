\documentclass{article}
\usepackage{amsmath}

\begin{document}

El algoritmo greedy toma los k maximos elementos del subconjunto, es decir en cada instancia toma el maximo elemento disponible del conjunto \\
Sea \(G\) la solución greedy para \( U, k \) dados, y sea \( O \) la solución óptima.


\begin{itemize}
    \item Si \( G > \text{O} \), es absurdo por definicion. Por lo tanto, \( G < \text{O} \) o \( G = \text{O} \).
    
    \item Si \( G = \text{O} \), es trivialmente cierto.
    
    \item Si \( G < \text{O} \): Sea \( i \) el primer índice donde \( O \) y \( G \) difieren. Por definición:
    
    \[ O = \{x_1, x_2, \dots, x_{i'}, x_{i' + 1}, \dots, x_n\} \]
    
    Dado que \( \forall {j'} \geq i: x_{j'} \leq x_i \) y  en particular (1) \( x_{i'} < x_i \), con \(x_i\) el iesimo elemento en G podemos considerar \( O^* = (O \cup \{x_i\}) - \{x_{i'}\} \)
		 \[ O^* = (O \cup \{x_i\}) - \{x_{i'}\} = \{x_1, x_2, \dots, x_{i-1}, x_i, x_{i' + 1}, \dots, x_n\} \]    
     y ahora la suma es \(  O^* >  O \) exactamente por \( x_i - x_{i'} \) y por (1) sabemos que \( x_i - x_{i'} > 0 \). Por lo tanto, es óptimo cambiar la decisión  i-esima tomada por \( O \) por la tomada por \( G \) . Esto se cumple para cualquier índice genérico, por lo tanto, greedy es óptimo.
    
    
\end{itemize}

\end{document}
