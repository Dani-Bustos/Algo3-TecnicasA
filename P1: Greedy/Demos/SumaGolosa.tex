\documentclass{article}
\usepackage{amsmath}

\begin{document}

El algoritmo greedy es tomar siempre los 2 mínimos es el paso óptimo, tomando en cuenta los pasos anteriores
Sea G el resultado del Greedy y O la solucion optima con A el multiconjunto numerico ordenado crecientemente

\[ A = \{a_1, a_2,\dots, a_n\} \] 
Sea \( P(N) \):= G = O para un conjunto de N elementos

\begin{itemize}
	
    
\item Si hay solo un elemento: tomar el  mínimo = óptimo.
    
    \item Supongamos que existe un \( N_0 \) tal que \( P(N) \) es correcto \(\forall n' \le n_0 \), 
    \\¿\( P(N_0) \rightarrow P(N_0 + 1) \)?  Todo esto que sigue aca esta mal\\
    
    \( O_{n_0+1} \ge O_{n0} + min(eleccion)  =^{HI} G_{n0} + min(eleccion)\)
    
    Si \(G_{n_0}\) \( < a_{n_0+1}  < a_{n_0+2} \) entonces el mejor es \( a_{n0+1} \) y \( a_{n_0+2} \).
    
    Si \(G_{n_0}\) \( > a_{n_0+1} \) y \(G_{n0}\) \( < a_{n_0+2} \) entonces el mejor es \( a_{n_0+1} \) y la instancia previa.
    
    Si \(G_{n_0}\) \( > a_{n_0+2}  >  a_{n_0+1} \), entonces el mejor es \( a_{n_0+1} \) y \( a_{n_0+2} \).
    
    Luego, para todos los casos, tomar siempre los 2 mínimos es la mejor opción.  Y \(G_{n_0 + 1} \) = \(O_{n_0+1}\), luego G es correcto 
    
\end{itemize}

\end{document}






